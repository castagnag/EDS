\documentclass[twoside,leqno,onecolumn]{article}  
\usepackage{DBSEncyArticle}

\begin{document}

\title{TITLE OF ENTRY}
\author{BYLINE \\ (Name of expert writing the entry, expert's affiliation)}

\date{}
\maketitle

\noindent
\begin{synonyms}
  \synonym{synonym1}
  \synonym{synonym2}
  \synonym{etc.} 
\end{synonyms}
~\\\\

\begin{definition}
	250 or fewer words defining the entry title.
\end{definition}
~\\\\
\begin{historicalBG}
	500 or fewer words describing when/why the concept or technique developed. Or related work.
\end{historicalBG}
~\\\\
\begin{scientific}
	Illustration and elaboration of the entry title definition, and outline the key points.
\end{scientific}
~\\\\
\begin{keyApp}
	Current and potential users and the motivation of studying this area.
\end{keyApp}
~\\\\
\begin{future}
	Optional. Open problems and discussions.
\end{future}
~\\\\
\begin{experimental}
	Optional. 
\end{experimental}
~\\\\
\begin{datasets}
	Optional.  
\end{datasets}
~\\\\
\begin{URL}
	Optional.  
\end{URL}
~\\\\
\begin{crossreference}
	Other topics in the Encyclopedia which may be of interest to the reader of this entry. It is encouraged to redirect the readers to an overview entry in the subject area.
\end{crossreference}
~\\
%\bibliographystyle{plain}
%\bibliography{sd}
%
%
%Either use bibliography below or use own bib file above
\begin{thebibliography}{8}
\bibitem{Aderem99}
  First bibliographic note
\bibitem{Janeway01}
  Second bibliographic note
\bibitem{Rosenberger03}
  Third bibliographic note
\end{thebibliography}


%Add definition entries here
\newpage

\begin{center}
\LARGE TITLE FOR DEFINITIONAL ENTRY
\vskip 1em
\large BYLINE \\ (Name of expert writing the entry, expert's affiliation) \\
\end{center}
\vskip 1.5em
\begin{synonyms}
  \synonym{synonym1}
  \synonym{synonym2}
  \synonym{etc.}
\end{synonyms}
~\\\\
\begin{definition}
	250 or fewer words defining the entry title.
\end{definition}
~\\\\
\begin{maintext}
	250 words outlining the key points.
\end{maintext}
~\\\\
\begin{crossreference}
	Listing of related entries, e.g. Remote Sensing, Spatial Data Quality, etc.
\end{crossreference}
~\\\\
\begin{reference}
	Optional. A list of 1-3 citations that give the reader a place to find more information.
\end{reference}

\end{document}
